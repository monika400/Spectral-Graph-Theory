\documentclass{article}
\usepackage[utf8]{inputenc}
\usepackage{amsmath}

\title{Wilson's Theorem}
\author{Bhavita}
\date{1st Feburary 2018}
\begin{document}

\maketitle

\section{Theorem}
Wilson's Theorem states that a natural number $ p>1 $ is a prime number if and only if 

       $ (p-1)!\equiv-1$   (mod p)
       
\section{Proof} 
We use the fact that if a polynomial $f(X)$ has integer coefficients, degree d and there are more than d values of $a\in\{0,1,2.....p-1\}$
with $f(a)\equiv0$(mod p) then all the coefficients of f are 
multiples of p.(It is essential that p be prime for this to be hold!).

We apply this observation to the polynomial \newline
  
  
  $f(X)=X^{p-1}-1-(X-1)(X-2)....(X-(p-1))=X^{p-1}-1-$$\prod_{p=1}^{k}(X-k)$\newline
  
  
If we substitute $X=a$ for $a\in\{1,2,3....,p-1\}$ in the product above,one of the factors become zero.Hence for $ a\in\{1,2...p-1\}$, \newline
  
  
  $f(a)=a^{p-1}-1 \equiv 1-1=0 $  (mod p) \newline
  
  
by Fermat's little theorem.The degree of f is less than p-1 as the coefficient of $X^{p-1}$is $1-1=0$.As there are p-1 solutions of $f(a)\equiv 0 (mod p) in \{1,2,3...p-1\}$,then all the coefficients 
of f are divisible by p. It follows that $f(0)\equiv0$ (mod p) that is \newline
  
  $0\equiv-1-\prod_{k=1}^{p-1}(-k)=-1-(-1)^{p-1}\prod_{k=1}^{p-1}(k)=-1-(p-1)!$ (mod p) \newline
  
On rearranging we get \newline
  
  $(p-1)!\equiv-1$ (mod p) \newline
  
   Hence proved!!
  

\end{document}
